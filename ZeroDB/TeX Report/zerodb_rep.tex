%%%%%%%%%%%%%%%%%%%%%%%%%%%%%%%%%%%%%%%%%
% Journal Article
% LaTeX Template
% Version 1.4 (15/5/16)
%
% This template has been downloaded from:
% http://www.LaTeXTemplates.com
%
% Original author:
% Frits Wenneker (http://www.howtotex.com) with extensive modifications by
% Vel (vel@LaTeXTemplates.com)
%
% License:
% CC BY-NC-SA 3.0 (http://creativecommons.org/licenses/by-nc-sa/3.0/)
%
%%%%%%%%%%%%%%%%%%%%%%%%%%%%%%%%%%%%%%%%%

%----------------------------------------------------------------------------------------
%	PACKAGES AND OTHER DOCUMENT CONFIGURATIONS
%----------------------------------------------------------------------------------------

\documentclass[twoside,twocolumn]{article}

\usepackage{blindtext} % Package to generate dummy text throughout this template 
\usepackage{amsmath}
\usepackage[sc]{mathpazo} % Use the Palatino font
\usepackage[T1]{fontenc} % Use 8-bit encoding that has 256 glyphs
\linespread{1.05} % Line spacing - Palatino needs more space between lines
\usepackage{microtype} % Slightly tweak font spacing for aesthetics
\usepackage{graphicx}
\usepackage[english]{babel} % Language hyphenation and typographical rules

\usepackage[hmarginratio=1:1,top=32mm,columnsep=20pt]{geometry} % Document margins
\usepackage[hang, small,labelfont=bf,up,textfont=it,up]{caption} % Custom captions under/above floats in tables or figures
\usepackage{booktabs} % Horizontal rules in tables

\usepackage{lettrine} % The lettrine is the first enlarged letter at the beginning of the text
\usepackage{listings}
\usepackage{enumitem} % Customized lists
\setlist[itemize]{noitemsep} % Make itemize lists more compact

\usepackage{abstract} % Allows abstract customization
\renewcommand{\abstractnamefont}{\normalfont\bfseries} % Set the "Abstract" text to bold
\renewcommand{\abstracttextfont}{\normalfont\small\itshape} % Set the abstract itself to small italic text
\usepackage{framed}
\usepackage{titlesec} % Allows customization of titles
\renewcommand\thesection{\Roman{section}} % Roman numerals for the sections
\renewcommand\thesubsection{\roman{subsection}} % roman numerals for subsections
\titleformat{\section}[block]{\large\scshape\centering}{\thesection.}{1em}{} % Change the look of the section titles
\titleformat{\subsection}[block]{\large}{\thesubsection.}{1em}{} % Change the look of the section titles

\usepackage{fancyhdr} % Headers and footers
\pagestyle{fancy} % All pages have headers and footers
\fancyhead{} % Blank out the default header
\fancyfoot{} % Blank out the default footer
%\fancyhead[C]{Running title $\bullet$ May 2016 $\bullet$ Vol. XXI, No. 1} % Custom header text
\fancyfoot[RO,LE]{\thepage} % Custom footer text

\usepackage{titling} % Customizing the title section

\usepackage{hyperref} % For hyperlinks in the PDF

%----------------------------------------------------------------------------------------
%	TITLE SECTION
%----------------------------------------------------------------------------------------

\setlength{\droptitle}{-4\baselineskip} % Move the title up

\pretitle{\begin{center}\Huge\bfseries} % Article title formatting
\posttitle{\end{center}} % Article title closing formatting
\title{Enterprise Grade Security and Data Protection for Big Data in the Cloud Using ZeroDB} % Article title
\author{%
\textsc{Shayan Fazeli - Sepideh Berangi - Foad Jafari} \\[1ex] % Your name
\normalsize Sharif University of Technology \\ % Your institution
\normalsize \href{mailto:shayan.fazeli@gmail.com}{shayan.fazeli@gmail.com} % Your email address
%\and % Uncomment if 2 authors are required, duplicate these 4 lines if more
%\textsc{Jane Smith}\thanks{Corresponding author} \\[1ex] % Second author's name
%\normalsize University of Utah \\ % Second author's institution
%\normalsize \href{mailto:jane@smith.com}{jane@smith.com} % Second author's email address
}
\date{\today} % Leave empty to omit a date
\renewcommand{\maketitlehookd}{%
\begin{abstract}
\noindent
The usage of big data analytics, a concept that nowadays everyone is familiar with, is growing continuously. Aside from all the advantages that this concept is offering, security is under debate. ZeroDB is a system like CryptDB that offers the benefits of database systems and the confidentiality of encryption schemas at the same time. As stated in zerodb's website: ``ZeroDB enables enterprises to use the Cloud for storage and computation while keeping encryption keys on-premise.''. In this approach we start working with this system from scratch.
\end{abstract}
}

%----------------------------------------------------------------------------------------

\begin{document}

% Print the title
\maketitle

%----------------------------------------------------------------------------------------
%	ARTICLE CONTENTS
%----------------------------------------------------------------------------------------

\section{Introduction}

\lettrine[nindent=0em,lines=3]{T} he confidentiality, authentication and integrity, are three main factors of a system's security. These factors must be addressed with encryption schemas that are secure to certain attacks. ZeroDB, uses these encryption schemas to provide us with a database system which has a layer of protection in front of it. In this project, we take a look at different aspects of this package.

\section{Installation}
In this part the installation of zerodb package is discussed in a step-by-step manner. Our version is installed on a specific linux distribution (ubuntu 16.04), but the installation procedure is pretty much the same for other linux distributions. First of all, python installer package known as ``pip'' must be collected and installed. This can be done using the following command:
\begin{framed}
sudo apt-get install python-pip
\end{framed}
Note that we used python2.7 as our environment for installing zerodb. Now pip is installed and ready to be used, therefore, zerodb python package can be installed using this command:
\begin{framed}
sudo pip install zerodb
\end{framed}
The version of zerodb which is available for pip will be detected automatically. When this is done, we need to clone the zerodb's repository from github. To do so, a new folder should be created. When in that new folder, one can insert this command to clone it:
\begin{framed}
git clone http://github.com/zerodb/\\zerodb-server.git
\end{framed}
After this phase, the requirements (like zerodb-server) must be installed. Change the working directory using ``cd'' command to the  ``demo'' folder inside zerodb's repository. In that, there is a file named ``requirements.txt'' that can be used as follows to install the necessary dependencies:
\begin{framed}
sudo pip install -r requirements.txt
\end{framed}
Now, run this command:
\begin{framed}
zerodb-manager init\_db
\end{framed}
Now the user properties of the manager can be set.
When this is done, zerodb is ready to use. Invoke zerodb using the following command:
\begin{framed}
zerodb-server
\end{framed}
Now, using python, this database system can be manipulated.
\section{Manipulation}
A table can be defined as follows:
\begin{framed}
\begin{lstlisting}[language=python]
class Employee(Model):
    name = Field()
    surname = Field()
    description = Text()
    salary = Field()

    def __repr__(self):
        return 
        "<%s %s who earns $%s>" 
        % (self.name, 
        self.surname, 
        self.salary)
\end{lstlisting}
\end{framed}

Using ``create.py'' script provided in demo folder, this table can be filled with many random values. Then, we head to the query part to see how we can ask for some information.
\section{Queries}
A video of submitting queries is sent along with this report. Supported queries are listed below:
\begin{framed}
Contains(index\_name, value)\\
Contains query.\\
Eq(index\_name, value)\\
    Equals query.\\
NotEq(index\_name, value)\\
    Not equal query\\
Gt(index\_name, value)\\
    Greater than query.\\
Lt(index\_name, value)\\
    Less than query.\\
Ge(index\_name, value)\\
    Greater (or equal) query.\\
Le(index\_name, value)\\
    Less (or equal) query.\\
DoesNotContain(index\_name, value)\\
    Does not contain query\\
Any(index\_name, value)\\
    Any of query.\\
NotAny(index\_name, value)\\
    Not any of query (ie, None of query)\\
All(index\_name, value)\\
    All query.\\
NotAll(index\_name, value)\\
    NotAll query.\\
InRange(index\_name, start, end, start\_exclusive=False, end\_exclusive=False)\\
    Index value falls within a range.\\
NotInRange(index\_name, start, end, start\_exclusive=False, end\_exclusive=False)\\
    Index value falls outside a range. 
 \end{framed}

Logic operators are listed below as well:
\begin{framed}

“\$and”: And\\
    Joins query clauses with a logical AND returns all documents that match the conditions of both clauses.\\
“\$or”: Or\\
    Joins query clauses with a logical OR returns all documents that match the conditions of either clause.\\
“\$not”: Not\\
    Inverts the effect of a query expression and returns documents that do not match the query expression. 
\end{framed}

And for field operations we have these operators:
\begin{framed}

“\$eq”: Eq\\
    Matches values that are equal to a specified value.\\
“\$ne”: NotEq\\
    Matches all values that are not equal to a specified value.\\
“\$lt”: Lt\\
    Matches values that are less than a specified value.\\
“\$lte”: Le\\
    Matches values that are less than or equal to a specified value.\\
“\$gt”: Gt\\
    Matches values that are greater than a specified value.\\
“\$gte”: Ge\\
    Matches values that are greater than or equal to a specified value.\\
“\$range”: InRange\\
    Matches values that fall within a specified range.\\
“\$nrange”: NotInRange\\
    Matches values that do not fall within a specified range.\\
“\$text”: Contains\\
    Performs text search for a specified value..\\
“\$ntext”: DoesNotContain\\
    Performs text search for the lack of a specified value.\\
“\$in”: Any\\
    Matches any of the values specified in an array.\\
“\$all”: All\\
    Matches arrays that contain all elements specified in the query.\\
“\$nany”: NotAny\\
    Matches none of the values specified in an array.\\
“\$nin”: NotAll\\
    Matches arrays that contain all elements specified in the query. \\
   \end{framed}


An script for testing simple queries is also written and is sent along.



%----------------------------------------------------------------------------------------
%	REFERENCE LIST
%----------------------------------------------------------------------------------------

\begin{thebibliography}{99} % Bibliography - this is intentionally simple in this template

\bibitem[ZeroDB]{dalaltriggs2005}
\newblock http://zerodb.com
 
\end{thebibliography}

%----------------------------------------------------------------------------------------

\end{document}
